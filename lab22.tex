\documentclass[a4paper, 12pt]{book}

\usepackage{geometry}
\usepackage[english, russian]{babel}
\usepackage[utf8]{inputenc}
\usepackage{wasysym}
\usepackage{amssymb}
\usepackage{amsfonts}
\usepackage{setspace}
\usepackage{tabto}


\geometry{left=3cm}
\geometry{right=4cm}
\geometry{top=4cm}
\geometry{bottom=2cm}

\setlength{\headheight}{0mm}
\setlength{\headsep}{0mm}
\setcounter{page}{584}

\begin{document}
    \begin{center}
        \begin{spacing}
        
                    ГЛ. VIII. ДИФФЕРЕНЦИАЛЬНОЕ ИСЧИСЛЕНИЕ
            \noindent\rule{\textwidth}{1pt}
	\end{spacing}
    \end{center}
    \par\textbf{2. Простейший вариант теоремы о неявной функции.} В этом
параграфе теорема о неявной функции будет получена очень наглядным, но не очень эффективным методом, приспособленным только к
случаю вещественнозначных функций вещественных переменных.
С другим, во многих отношениях более предпочтительным способом получения этой теоремы, как и с более детальным анализом ее структуры,
читатель сможет познакомится в главе X (часть II), а также в задаче 4,
помещенной в конце параграфа..
    \parСледующее утверждение является простейшим вариантом теоремы
о неявной функции.
    \par\textbf{Утверждение 1.} \textit{Если функция $F \to U(x_0, y_0) \to \mathbb{R}$, определенная в
окрестности $U(x_0, y_0)$ точки $(x_0, y_0) \in R^2$, такова, что}
    \begin{spacing}{1.6}
        \par \textit{$1^\circ$ $F \in C^(p) (U;\mathbb{R})$, где $p >= 1$,}
        \par \textit{$2^\circ$ $F(x_0, y_0) = 0$,}
        \par \textit{$3^\circ$ $F'_y(x_0, y_0) \neq 0$,}
    \end{spacing}
    \par \textit{то существуют двумерный промежуток $I = I_x \times I_y$, где}
    $$I_x = \{x \in \mathbb{R} \mid |x - x_0| < a\}, \;\;\;\;\; I_y=\{ y \in \mathbb{R} \mid |y-y_0| < \beta \}$$
    \par \textit{являющийся содержащейся в $U(x_0, y_0)$ окрестностью точки $(x_0, y_0)$, и
такая функция $f \in C^(p) (I_x; I_y)$, что для любой точки $(x, y) \in I_x \times I_y$}
    \begin{flushright} 
    $F(x, y) = 0 \Leftrightarrow y = f(x),  \quad \quad \quad \quad \quad \quad \quad \quad \quad \quad \textrm{(4)}$
    \end{flushright}
    \par \textit{причем производная функции $y = f(x)$ в точках $x \in I_x$ может быть
вычислена по формуле}
    \begin{flushright} 
    $f'(x) = -[F'_y(x, f(x))]^1[F'_x(x, f(x))]. \quad \quad \quad \quad \quad \quad \quad \quad \textrm{(5)}$
    \end{flushright}
    \par Прежде чем приступить к доказательству, дадим несколько возможных переформулировок заключительного соотношения (4), которые должны заодно прояснить смысл самого этого соотношения.
    \par Утверждение 1 говорит о том, что при условиях $1^\circ, 2^\circ, 3^\circ$ порция множества, определяемого соотношением $F(x, y) = 0$, попавшая в окрестность $I = I_x \times I_y$ точки $(x_0, y_0)$, является графиком некоторой функции $f : Ix \to Iy$ класса $C^(p)(I_x; I_y)$.
    \par Иначе можно сказать, что в пределах окрестности $I$ точки $(x_0, y_0)$ уравнение $F(x, y) = 0$ однозначно разрешимо относительно $y$, а функция $y = f(x)$ является этим решением, т. е. $F(x, f(x)) \equiv 0$ на $I_x$.

    \newpage
 
    \begin{center}
        \begin{spacing}
            \S5. ТЕОРЕМА О НЕЯВНОЙ ФУНКЦИИ 
            \noindent\rule{\textwidth}{1pt}
	\end{spacing}
    \end{center}
    \par Отсюда в свою очередь следует, что если $y = \hat{f}(x)$ — функция, определенная на $I_x$, про которую известно, что она удовлетворяет соотношению $F(x, \hat{f}(x)) \equiv 0$ на $I_x$ и что $\hat{f}(x_0) = y_0$, то при условии непрерывности этой функции в точке $x_0 \in I_x$ можно утверждать, что найдется окрестность $\Delta \subset I_x$ точки $x_0$ такая, что $\hat{f}(\Delta) \subset I_y$ и тогда $\hat{f}(x) \equiv f(x)$ при $x \in \Delta$.
    \par Без предположения непрерывности функции $\hat{f}$ в точке $x_0$ и условия $\hat{f}(x_0) = y_0$ последнее заключение могло бы оказаться неправильным, что видно на уже разобранном выше примере с окружностью. Теперь докажем утверждение 1.
    \par \blacktriangleleft Пусть для определенности $F′_y(x_0, y_0)>0$. Поскольку $F \in C^{(1)}(U; \mathbb{R}),то F′_y(x, y) > 0$ также в некоторой окрестности точки (x0, y0). Чтобы невводить новых обозначений, без ограничения общности можно считать,что $F′_y(x, y) > 0$ в любой точке исходной окрестности $U(x_0, y_0)$.
    \par Более того, уменьшая, если нужно, окрестность $U(x_0, y_0)$, можно
считать ее кругом некоторого радиуса $r = 2\beta > 0$ с центром в точке $(x0, y0)$.
    \par Поскольку $F′_y(x, y) > 0$ в $U$, то функция $F(x_0, y)$ от $y$ определена имонотонно возрастает на отрезке $y_0-\beta \leq y \leq y_0 + \beta$, следовательно,
    $$F(x_0, y_0 - \beta) < F(x_0, y_0) = 0 < F(x_0, y_0 + \beta)$$
    \par В силу непрерывности функции $F$ в $U$, найдется положительное число $\alpha < \beta$ такое, что при $|x − x_0| \leq \alpha$ будут выполнены соотношения
    $$F(x, y_0 - \beta) < 0 < F(x, y_0 + \beta)$$
    \par Покажем теперь, что прямоугольник $I = I_x × I_y$, где
    $$I_x = \{ x \in \mathbb{R} \mid |x - x_0| < \alpha \}, \;\;\;\; I_y = \{ y \in \mathbb{R} \mid |y - y_0| < \beta \}$$
    \par является искомым двумерным промежутком, в котором выполняется соотношение (4).
    \par При каждом $x \in I_x$ фиксируем вертикальный отрезок с концами $(x, y_0 − \beta), (x, y_0 + \beta)$. Рассматривая на нем $F(x, y)$ как функцию от $y$, мы получаем строго возрастающую непрерывную функцию, принимающую значения разных знаков на концах отрезка. Следовательно, при $x \in I_x$ найдется единственная точка $y(x) \in I_y$ такая, что $F(x, y(x)) = 0$. Полагая $y(x) = f(x)$, мы приходим к соотношению (4).

\end{document}